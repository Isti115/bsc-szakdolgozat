\section{Funkcionális követelmények}

\subsection{Hőtérképes megjelenítés}

A drónokról érkező (numerikus) adatok megjelenítése színes hőtérkép segítségével.
A jelölő pontok méretei, a köztük lévő távolság, valamint a színskála személyre szabható (lineáris / logaritmikus, a színek határai), továbbá bekapcsolható a rácspontokhoz igazítás, ami egy négyzetrács csúcsaira illeszti a kapott értékeket.

\begin {itemize}
  \item \textit{Ha} egy hőtérkép réteg aktív és fel van iratkozva egy adott csatornára egy drónon, \textit{amikor} új adat érkezik az érzékelőtől \textit{akkor} az megjelenik a térképen.
  \item \textit{Ha} már van adat kirajzolva a térképen és aktív az automatikus skálázás, \textit{amikor} egy új szélsőséges érték érkezik, ami a jelenlegi minimum és maximum által meghatározott tartományom kívül esik, \textit{akkor} a korlátok mellett a már meglévő pontok színei is frissülnek.
\end {itemize}


\subsection{Térkép állapotainak (pozíció, forgatás, nagyítás) tárolása és betöltése}

Egy panel, amelyen elmenthetőek, szerkeszthetőek és visszatölthetőek adott nézetek. A tárolt adatok: a középpont koordinátái, az elforgatás szöge, a nagyítás mértéke, valamint a nézet neve.

\begin {itemize}
  \item \textit{Ha} a felhasználó a kívánt állapotba mozgatta, forgatta és nagyította a térképet, \textit{amikor} a felhasználó rákattint az "Add a new location" lehetőségre, \textit{akkor} az aktuális pozíció, elfordulás és nagyítás rögzítésre kerül és a felugró dialógusablakban felhasználónak lehetősége van átnevezni a létrehozni kívánt bejegyzést és elmenteni, illetve elvetni azt.
  \item \textit{Ha} a felhasználó már előzőleg elmentett egy térképállapotot, \textit{amikor} a felhasználó kiválasztja ezt a térképállapotot, \textit{akkor} a térkép abba az állapotba mozog, forog és nagyítódik.
\end {itemize}


\subsection{Objektumok betöltése GeoJSON formátumból}

Különböző alakzatok megjelenítése a térképen, például repülési zóna vizualizálására, tereptárgyak kijelölésére.

\begin {itemize}
  \item \textit{Ha} a felhasználó létrehozott egy GeoJSON réteget és a felkínált szövegdobozba érvényes GeoJSON adatokat másolt, \textit{amikor} a felhasználó megnyomja az "IMPORT GEOJSON" gombot, \textit{akkor} az alkalmazás megjeleníti a leírt objektumokat a térképen.
  \item \textit{Ha} a felhasználó invalid értéket másolt a szövegdobozba, \textit{amikor} megnyomja az "IMPORT GEOJSON" gombot \textit{akkor} a térképen az előzőleg kirajzolt objektumok maradnak rajta.
\end {itemize}


\subsection{Parancsok kiadása drónoknak menüsorról és jobb egérgombra felugró ablakból}

Felszállás, leszállás, visszatérés illetve egyéb parancsok és üzenetek küldése drónoknak. Ha vannak kiválasztott egységek, akkor a menüsoron lévő akciók az összes jelenleg kijelölt kopterre hatással vannak, egy kopterre jobb egérgombbal kattintva pedig egy felugró ablakban válnak elérhetővé ugyanezek a lehetőségek arra az adott drónra vonatkozóan.

\begin {itemize}
  \item \textit{Ha} vannak kijelölve drónok, \textit{amikor} a felhasználó jobb egérgombbal kattint a térképen egy üres területen, \textit{akkor} egy felugró menü lehetőséget biztosít parancs kiadására a kijelölt drónoknak.
  \item \textit{Ha} vannak kijelölve drónok, \textit{amikor} a felhasználó jobb egérgombbal kattint a térképen egy drón felett, \textit{akkor} a drón hozzáadódik a kijelöléshez és egy felugró menü lehetőséget biztosít parancs kiadására a kijelölt drónoknak.
  \item \textit{Ha} vannak kijelölve drónok, \textit{amikor} a felhasználó kiválaszt egy parancsot az eszköztárról vagy pedig a felugró menüből, \textit{akkor} a drónok végrehajtják az adott parancsot.
\end {itemize}


\subsection{Állapotüzenetek (információk, figyelmeztetések, hibák) kijelzésére alkalmas panel}

Napló vezetése az alkalmazás különböző üzeneteiről, valamint ennek megjelenítése egy sorba rendezhető, szűrhető táblázatban.

\begin {itemize}
  \item \textit{Ha} egy esemény naplóbejegyzést eredményez, \textit{amikor} lezajlik az adott esemény, \textit{akkor} az általa készített bejegyzés eltárolódik a naplóban.
  \item \textit{Ha} aktív a felületen a naplóbejegyzések kijelzésére alkalmas panel, \textit{amikor} új bejegyzés érkezik, \textit{akkor} az azonnal automatikusan megjelenik a táblázatban.
  \item \textit{Ha} aktív a felületen a naplóbejegyzések kijelzésére alkalmas panel, \textit{amikor} a felhasználó megnyomja a rendezés gombot valamelyik oszlop fejsorában, \textit{akkor} a sorok az ahhoz az oszlophoz tartozó értékük szerinti sorrendben jelennek meg.
  \item \textit{Ha} aktív a felületen a naplóbejegyzések kijelzésére alkalmas panel, \textit{amikor} a felhasználó szűrést alkalmaz valamelyik oszlopra, \textit{akkor} a táblázatban csak a szűrési feltételnek eleget tevő sorok maradnak láthatóak.
\end {itemize}


\subsection{Drónok színkódolása predikátumok alapján}

A drónok alapértelmezetten feketén jelennek meg, további színekhez pedig megadhatóak JavaScript függvénytörzsek, melyek paraméterül egy drón nevét kapják.
Amennyiben egy bizonyos színhez megadott feltétel, valamely azonosítót kapva paraméterként, igazra értékelődik ki, akkor az adott drón a térképen olyan színnel jelenik meg.
Például \verb|blue: 'Number(id) < 10'| esetén az 5-ös azonosítójú drón kék lesz, a 15-ös azonban nem.

\begin {itemize}
  \item \textit{Ha} egy adott drón azonosítója egyik színhez megadott predikátumnak sem tesz eleget, \textit{amikor} a térképre rajzolódik, \textit{akkor} a hozzá tartozó piktogram színe fekete.
  \item \textit{Ha} a felhasználó bevisz egy érvényes predikátumot az egyik színhez tartozó mezőbe, \textit{amikor} megnyomja az "APPLY CHANGES" gombbot, \textit{akkor} az olyan drónok, amelyek megfelelnek ennek a predikátumnak (és semelyik másiknak sem), az adott színű ikonnal jelennek meg a térképen.
\end {itemize}


\subsection{Drónok részletes aktuális állapotát mutató panel}

Egy táblázat, amiben minden drónhoz egy-egy sor tartozik, melyben a rá vonatkozó részletes információk (akkufeszültség, aktuálisan futó algoritmus / koreográfia, stb.) láthatóak.

\begin {itemize}
  \item \textit{Ha} a drónokról ismert aktuális információk megjelenítésére szolgáló panel aktív, \textit{amikor} új adat érkezik valamelyik drónról \textit{akkor} a táblázat megfelelő sora frissül.
\end {itemize}


\subsection{Kliens állomás pozíciójának megjelenítése a térképen geolokáció alapján}

Amennyiben a kliens eszköz rendelkezik geolokáció szolgáltatással, akkor a megfelelő réteg hozzáadását követően megjelenik az aktuális helyzete és iránya a térképen.
A funkció bekapcsolását követően a jelölő pozíciója és állása folyamatosan frissül az eszköz által rendelkezésre bocsátott információk alapján.

\begin {itemize}
  \item \textit{Ha} az alkalmazást futtató eszköz támogatja a geolokáció szolgáltatást, \textit{amikor} a felhasználó létrehoz egy "Own location" réteget, \textit{akkor} a jelölő felkerül a térkép megfelelő pontjára.
  \item \textit{Ha} van aktív "Own location" típusú réteg és az eszköz támogatja a geolokáció lekérdezését, \textit{amikor} a felhasználó pozíciója megváltozik, \textit{akkor} a jelölő is követi az elmozdulást.
  \item \textit{Ha} van aktív "Own location" típusú réteg és az eszköz a felhasználó pozícióján kívül orientációra vonatkozó adatokat is elérhetővé tesz, \textit{amikor} az eszköz iránya megváltozik, \textit{akkor} a jelölő is követi az elfordulást.
\end {itemize}


\subsection{Billentyűkombinációkkal történő vezérlés}

Az alkalmazás bizonyos funkcióinak elérése a billentyűzet segítségével, például kijelölés kezelése, parancs küldése drónoknak, dialógusablakok megnyitása.
Súgó elérése a billentyűkombinációk megtekintésére.

\begin {itemize}
  \item \textit{Ha} semmilyen beviteli mező sem aktív, \textit{amikor} a felhasználó megnyomja a "?" karakter begépelésére szolgáló billentyűkombinációt, \textit{akkor} megjelenik az összes elérhető gyorsbillentyűt tartalmazó táblázat.
  \item \textit{Ha} semmilyen beviteli mező sem aktív, \textit{amikor} a felhasználó megnyom egyet az elérhető gyorsbillentyűk közül, \textit{akkor} végrehajtódik a hozzá tartozó parancs.
\end {itemize}
