\subsection{Felhasznált főbb könyvtárak}

\begin{itemize}
  \item \textbf{React}: A központi állapot nézetté történő leképezésére, a HTML
  elemek fölött használt állapottal rendelkező komponensek absztrakciójának
  megteremtésére. (A hagyományos MVC architektúrában leginkább a view szerepét
  tölti be.) \\
  https://reactjs.org/

  \item \textbf{Redux}: Az alkalmazás központi állapotának tárolásához és
  kezeléséhez. (A tradicionális MVC architektúrából leginkább a modell és a
  kontroller kombinációjának feleltethető meg.) \\
  https://redux.js.org/

  \item \textbf{OpenLayers}: A térképes megjelenítéshez. \\
  https://openlayers.org/

  \item \textbf{ol-react}: Az OpenLayers könyvtár által nyújtott funkciók React
  komponensekként való kezeléséhez. \\
  https://www.npmjs.com/package/ol-react

  \item \textbf{Golden Layout}: A dinamikus, felhasználó által személyre szabható
  felület kialakítására. \\
  https://golden-layout.com/

  \item \textbf{Material UI}: A felhasználói felületen megjelenő elemek
  kinézetéhez. \\
  http://www.material-ui.com/ | https://material-ui-next.com/

  \item \textbf{mini-signals}: Az alkalmazás komponensei közötti üzenetküldésre.
  \\
  https://github.com/Hypercubed/mini-signals

  \item \textbf{Electron}: Az önállóan futtatható becsomagolt állomány
  létrehozásához. \\
  https://electronjs.org/
\end{itemize}

\section{Háttér modell}

\subsection{A Redux működése}

Mint az már korábban említésre került, a modell és a kontroller szerepét
egyaránt a Redux csomag tölti be. Ennek működése azon alapul, hogy a háttérben
több kisebb független modell összekapcsolásával jön létre egy nagyobb állapot,
melyeknek a módosítása kizárólag előre megadott akciók segítségével történik,
így garantálva a konzisztenciát.

// TODO: Könytárak részletesebb kifejtése

\noindent Perzisztencia céljából ezen állapot bizonyos részei kerülnek adott
időközönként automatikusan mentésre a böngésző által felkínált
\verb|localStorage| API segítségével, az alkalmazás megnyitásakor pedig innen
töltődnek be, így például a beállított színek és az elmentett térkép állapotok
megmaradnak az oldal frissítése után is.
