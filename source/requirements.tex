\section{Követelmények}

Szerver oldalon tehát szükséges egy számítógép, amely hardware tekintetében
rendelkezik Wi-Fi rádióval, vagy csatlakoztatva van hozzá XBee adó-vevő.
Szoftver területén megfelelő Wi-Fi vagy XBee driver, illetve a szerver
kódjának futtatásához \verb|python 3.6| a követelmény.

A kliens oldal kódjának lefordítása \verb|nodejs| által zajlik (v8.9.0-es
verzión tesztelve), a függőségek kezelését az \verb|npm| csomagkezelő végzi,
továbbá a módosított csomagok github-ról történő installációja miatt \verb|git|
kliens is szükséges. Az elkészült állomány egy webszerverről kiszolgálva egy
böngészőből érhető el, vagy pedig \verb|electron| segítségével becsomagolva
önállóan futtatható programként.

A forráskód birtokában, annak gyökérmappájában kiadva az \verb|"npm install"|
parancsot, telepíthetőek a függőségek. Ezt követően az \verb|"npm start"|
utasítás elvégzi a JavaScript kód kötegelését és kiszolgálja a kliens
alkalmazást alkotó tartalmakat egy webszerver segítségével, amely egy
böngészőből alapértelmezetten a http://localhost:8080/ címen érhető el.

A program önállóan futó alkalmazásként történő használatának előfeltétele, hogy
az electron csomag globálisan telepítve legyen. Amennyiben ez teljesül, akkor az
\verb|"npm run start:electron"| parancs indítja el saját külön ablakban a
klienst. A terjesztésre szánt futtatható állomány előállításához szükség van még
ezeken felül az \verb|"electron-builder"| meglétére. Ennek a hordozható file-nak
a létrehozását az \verb|"npm run dist"| utasítás kiadásával lehet kezdeményezni.

