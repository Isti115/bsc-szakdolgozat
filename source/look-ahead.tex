\section{Kitekintés}

A jövőre nézve a projekt továbbfejlesztésének nagyon sok iránya képzelhető el.

Ezek közül néhány példát hozva megemlíteném a más gyártótól származó drónokkal
való kommunikáció megvalósításának lehetőségét. Az kereskedelmi forgalomban
elterjedtebb márkájú kopterek közül jónéhány rendelkezik nyíltan közzétett
specifikációval a rádiós kommunikációját illetően, továbbá sokszor azoknak a
drónoknak is megtalálható a visszafejtett protokolljuk, amelyek nem rendelkeznek
ilyen jellegű publikus dokumendációval.

Egy másik lehetőség azon az észrevételen alapul, hogy a mai világban már
igencsak elterjedt mobil eszközöknek köszönhetően sokaknak a zsebében lapul
webböngészővel és kellő számítási kapacitással rendelkező számítógép, ami így
tulajdonképpen már jelenlegi állapotában is alkalmas a szoftver futtatására,
viszont a szóban forgó eszköz képernyőjének méretéből és a felület
elrendezéséből adódóan nem nyújt kényelmes felhasználói élményt. Érdemes lehet
tehát létrehozni egy külön mobil verziót, amely mondjuk egyszerre csak egy panel
megjelenítését végzi. Ehhez a programot alkotó React komponensek modularitásának
köszönhetően nagyon nagy részét újra lehetne hasznosítani a forráskódnak. Sok
kisebb képernyőn megjelenő panelből akár összeállítható lenne egy nagyobb
vezérlő felület is, mint mondjuk egyfajta műszerfal.

Egy várhatóan hamarosan elkészülő, de a jelen szakdolgozat íródásának
időpontjában még nem elérhető funkció továbbá a drónok telepfeszültségének
jelzése a térképen, így egy leszállás után könnyebben meg lehetne találni, hogy
melyik kopterek akkumulátorait szükséges lecserélni a következő felszállás
előtt.
