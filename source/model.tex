\section{Háttér modell}

Mint az már korábban említésre került, a modell és a kontroller szerepét
egyaránt a Redux csomag tölti be. A háttérben több kisebb független modell
összekapcsolásával jön létre egy nagyobb állapot, melyeknek a módosítása
kizárólag előre megadott akciók segítségével történik, így garantálva a
konzisztenciát. Ezek közül az program általam megvalósított részeihez a
következők relevánsak:

\begin{itemize}

  \item \textbf{layers.uavs.parameters.colorPredicates}

    A drónokhoz tartozó réteg paraméterei közül a színkódolásért felelős
    predikátumok tárolása is a Redux store-ban történik. Erre egy olyan objektum
    szolgál, melynek kulcsai az elérhető színek, az ezekhez rendelt értékek
    pedig a predikátum függvények törzsei.

    \textit{Módosítására szolgáló műveletek:}

    \begin{itemize}
      \item SET\_LAYER\_PARAMETER\_BY\_ID \\
        A rétegek paramétereinek beállítására szolgáló általános akció.
        Amennyiben a koptereket megjelenítő réteg azonosítójával és a
        "colorPredicates" paraméterrel kerül meghívásra, akkor az argumentumként
        adott predikátumokat eltárolja és érvénybe lépteti.
    \end{itemize}

  \item \textbf{log}

    Az eseménynapló bejegyzéseit tartalmazó állapot. Adattípus tekintetében egy
    listaként valósul meg, amiben időrendi sorrendben szerepelnek olyan
    objektumok, melyek egy adott üzenetnek tartalmazzák az azonosítóját,
    időbélyegét, kritikussági szintjét és szövegét.

    \textit{Módosítására szolgáló műveletek:}

    \begin{itemize}
      \item ADD\_LOG\_ITEM \\
        A kapott paraméterek alapján létrehoz egy új naplóbejegyzést, amit ellát
        az aktuális időbélyeggel és hozzáadja a tárolásra szolgáló listához.

      \item DELETE\_LOG\_ITEM \\
        Törli a megadott azonosító által jelölt naplóbejegyzést.

      \item CLEAR\_LOG\_ITEMS \\
        Kiüríti a naplót, törölve ezzel az összes bejegyzést.

      \item UPDATE\_LOG\_PANEL\_VISIBILITY \\
        A naplóbejegyzések megjelenítésére szolgáló lista láthatóságának
        változásakor kezdeményezett akció. Paraméterként megkapja, hogy éppen
        megnyílt, vagy pedig bezáródott a panel.
        A bal oldali sávon a panel ikonja mellett megjelenő, olvasatlan kritikus
        üzeneteket jelző színes jelölőhöz szükséges.

    \end{itemize}

  \item \textbf{saved-locations}

    Az eltárolt pozíciók adatait (név, középpont, nagyítás, elforgatás)
    tartalmazó objektumokból álló állapot.

    \textit{Módosítására szolgáló műveletek:}

    \begin{itemize}
      \item UPDATE\_SAVED\_LOCATION \verb|(savedLocation)| \\
        Meglévő elmentett pozíció módosítása, vagy ha a paraméterként kapott
        pozíció az éppen aktuális pozíció, akkor új bejegyzés létrehozása.

      \item DELETE\_SAVED\_LOCATION (savedLocationId) \\
        Egy előzőleg elmentett pozíció törlése annak azonosítója alapján.
    \end{itemize}

\end{itemize}

\noindent Perzisztencia céljából ezen állapot bizonyos részei kerülnek adott időközönként
automatikusan mentésre a böngésző által felkínált \verb|localStorage| API
segítségével, az alkalmazás megnyitásakor pedig innen töltődnek be, így például
a beállított színek és az elmentett térkép állapotok megmaradnak az oldal
frissítése után is.
