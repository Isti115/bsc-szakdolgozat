\section{A szoftver architektúrája}

\subsection{Futási környezet}

A program webes technológiákon alapul, így működtethető böngészőből, illetve böngészőtől független önálló becsomagolt állományból is.

\subsection{Felhasznált könyvtárak}

\begin{itemize}
  \item Electron: Az önlállóan futtatható állomány létrehozásához.

  \item Redux: Az alkalmazás állapotának tárolásához.
  \item Golden Layout: A dinamius, felhasználó által személyre szabható felület kialakítására.

  \item Material UI: A felhasználói felületen megjelenő elemekhez.

  \item OpenLayers: A térképes megjelenítéshez.

  \item React: Minden másra? :D
\end{itemize}

\subsection{A megvalósított osztályok}

\subsubsection{HeatmapLayerSettingsPresentation}
\subsubsection{HeatmapVectorSource}
\subsubsection{HeatmapLayerPresentation}

\subsubsection{MapReferenceRequestHandler}
\subsubsection{MapViewManager}
\subsubsection{SavedLocationEditorDialog}
\subsubsection{SavedLocationList}

\subsubsection{GeoJSONLayerSettingsPresentation}
\subsubsection{GeoJSONVectorSource}
\subsubsection{GeoJSONLayerPresentation}

\subsubsection{MapToolbar}
\subsubsection{ContextMenu}

\subsubsection{FilterableSortableTable}
\subsubsection{LogPanel}

A naplóbejegyzéseket tartalmazó panel.

\subsubsection{OwnLocationLayerSettingsPresentation}
\subsubsection{OwnLocationVectorSource}
\subsubsection{OwnLocationLayerPresentation}

A felhasználó geolokációjának térképen való megjelenítéséért felelős osztály.

\subsubsection{HotkeyHandler}

A billentyűparancsok érzékeléséért és a hozzájuk rendelt feladatok végrehajtásáért felelős osztály.
