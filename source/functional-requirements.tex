\section{Funkcionális követelmények}

\subsection{Hőtérképes megjelenítés}

A drónokról érkező (numerikus) adatok megjelenítése színes hőtérkép segítségével.
A jelölő pontok méretei, a köztük lévő távolság, valamint a színskála személyre szabható (lineáris / logaritmikus, a színek határai), továbbá bekapcsolható a rácspontokhoz igazítás, ami egy négyzetrács csúcsaira illeszti a kapott értékeket.

\begin {itemize}
  \item \textit{Given} that a heatmap layer is active and is subscribed to a channel on a drone \textit{when} new data arrives from the sensor \textit{then} it is displayed on the map.
  \item \textit{Given} that there is already data, \textit{when} a new value is received, that is below the current minimum or above the current maximum, \textit{then} the color scale is adjused accordingly.
\end {itemize}


\subsection{Térkép állapotainak (pozíció, forgatás, nagyítás) tárolása és betöltése}

Egy panel, amelyen elmenthetőek, szerkeszthetőek és visszatölthetőek adott nézetek. A tárolt adatok: a középpont koordinátái, az elforgatás szöge, a nagyítás mértéke, valmint a nézet neve.

\begin {itemize}
  \item \textit{Given} that the user has moved, rotated and zoomed the map to the desired state \textit{when} the user presses the "Add a new location" button \textit{then} the actual position, rotation and zoom state is stored and the user is prompted for a name.
  \item \textit{Given} that the user has previously stored a location \textit{when} the user select that location \textit{then} the map view is moved, rotated and zoomed to match the stored data.
\end {itemize}


\subsection{Objektumok betöltése GeoJSON formátumból}

Különböző alakzatok megjelenítése a térképen, például repülési zóna vizualizálására, tereptárgyak kijelölésére.

\begin {itemize}
  \item \textit{Given} that the user has created a new layer, \textit{when} the user has selected the GeoJSON layer type, \textit{then} the application displays an input field.
  \item \textit{Given} that the input is valid GeoJSON, \textit{when} the user clicks the import button, \textit{then} the objects described in the input appear on the map.
\end {itemize}


\subsection{Parancsok kiadása drónoknak menüsorról és jobb egérgombra felugró ablakból}

Felszállás, leszállás, viszatérés illetve egyéb parancsok és üzenetek küldése drónoknak. Ha vannak kiválasztott egységek, akkor a menüsoron lévő akciók az összes jelenleg kijelölt kopterre hatással vannak, egy kopterre jobb egérgombbal kattintva pedig egy felugró ablakban válnak elérhetővé ugyanezek a lehetőségek arra az adott drónra vonatkozóan.

\begin {itemize}
  \item \textit{Given} that no drone is currently selected \textit{when} the user presses the right mouse button over a single drone, \textit{then} a popup menu appears that lets the user issue commands to that UAV.
\end {itemize}


\subsection{Állapotüzenetek (információk, figyelmeztetések, hibák) kijelzésére alkalmas panel}

Napló vezetése az alkalmazás különböző üzeneteiről, valamint ennek megjelenítése egy sorbarendezhető, szűrhető táblázatban.

\begin {itemize}
  \item \textit{Given} that an event produces a log entry, \textit{when} the message arrives, \textit{then} it is stored in the log.
  \item \textit{Given} that the log panel is active, \textit{when} the user clicks the order button on one of the column headers, \textit{then} the rows are ordered according to that column.
  \item \textit{Given} that the log panel is active, \textit{when} the user filters one of the columns, \textit{then} only those rows are displayed that match the criteria.
\end {itemize}


\subsection{Drónok színkódolása predikátumok alapján}

A drónok alapértelmezetten feketén jelennek meg, további színekhez pedig megadhatóak javascript függvénytörzsek, melyek paraméterül egy drón nevét kapják.
Amennyiben egy bizonyos színhez megadott feltétel valamely azonosítót kapva paraméterként igazra értékelődik ki, akkor az adott drón a térképen olyan színnel jelenik meg.
Például \verb|blue: 'Number(id) < 10'| esetén az 5-ös azonosítójú drón kék lesz, a 15-ös azonban nem.

\begin {itemize}
  \item \textit{Given} that there are no matching predicates, \textit{when} a drone is rendered onto the map, \textit{then} it's color is black.
  \item \textit{Given} that the user has entered a valid predicate for color A, \textit{when} the predicate evaluates to a truthy value for drone B, \textit{then} drone B is displayed in color A.
\end {itemize}


\subsection{Drónok részletes aktuális állapotát mutató panel}

Egy táblázat, amiben minden drónhoz egy-egy sor tartozik, melyben a rá vonatkozó részletes információk (akkufeszültség, aktuálisan futó algoritmus / koreográfia, stb.) láthatóak.

\begin {itemize}
  \item \textit{Given} that the information panel is active, \textit{when} the user clicks the sort button \textit{then} the rows are ordered with respect to the appropriate column.
  \item \textit{Given} that the user has selected a row with the arrow keys, \textit{when} the user presses enter \textit{then} the message dialog of the selected UAV appears.
\end {itemize}


\subsection{Kliens állomás pozíciójának megjelenítése a térképen geolokáció alapján}

Amennyiben a kliens eszköz rendelkezik geolokáció szolgáltatással, akkor a megfelelő réteg hozzáadását követően megjelenik az aktuális helyzete és iránya a térképen.
A funkció bekapcsolását követően a jelölő pozíciója és állása folyamatosan frissül az eszköz által rendelkezésre bocsátott információk alapján.

\begin {itemize}
  \item \textit{Given} that the device in use supports geolocation \textit{when} the user adds the geolocation layer, \textit{then} the marker appears.
  \item \textit{Given} that the geolocation layer has been added, \textit{when} the user moves, \textit{then} the marker moves as well.
  \item \textit{Given} that the geolocation layer has been added, \textit{when} the user turns, \textit{then} the marker turns as well.
  \item \textit{Given} that the geolocation layer has been added, \textit{when} the user deletes the geolocation layer, \textit{then} the marker disappears.
\end {itemize}


\subsection{Billentyűkombinációkkal történő vezérlés}

Az alkalmazás bizonyos funkcióinak elérése a billentyűzet segítségével, például kijelölés kezelése, parancs küldése drónoknak, dialógusablakok megnyitása.
Súgó elérése a billentyűkombinációk megtekintésére.

\begin {itemize}
  \item \textit{Given} that no input field is active, \textit{when} the user presses the ? key, \textit{then} the available hotkeys are displayed.
  \item \textit{Given} that no input field is active, \textit{when} the user presses a hotkey, \textit{then} the desired action is performed.
\end {itemize}
